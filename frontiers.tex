%%%%%%%%%%%%%%%%%%%%%%%%%%%%%%%%%%%%%%%%%%%%%%%%%%%%%%%%%%%%%%%%%%%%%%%%%%%%%%%%%%%%%%%%%%%%%%%%%%%%%%%%%%%%%%%%%%%%%%%%%%%%%%%%%%%%%%%%
% This is just a template to use when submitting manuscripts to Frontiers, it is not mandatory to use frontiers.cls nor frontiers.tex  %
%%%%%%%%%%%%%%%%%%%%%%%%%%%%%%%%%%%%%%%%%%%%%%%%%%%%%%%%%%%%%%%%%%%%%%%%%%%%%%%%%%%%%%%%%%%%%%%%%%%%%%%%%%%%%%%%%%%%%%%%%%%%%%%%%%%%%%%%

\documentclass{frontiersSCNS} % for Science articles
%\documentclass{frontiersMED} % for Medicine articles

\usepackage{url}
\usepackage{lineno}
\usepackage{todonotes}

\newcommand{\alex}[1]{\todo[inline, color=green!40]{#1}}



\linenumbers


\copyrightyear{}
\pubyear{}
%\onecolumn
%%% write here for which journal %%%
\def\journal{Neurosciences}
\def\DOI{}
\def\articleType{}
\def\citing{\color{darkgray}\cite}
\def\keyFont{\fontsize{6}{11}\helveticabold }
\def\firstAuthorLast{Alexandre Abraham {et~al}} %use et al only if is more than 1 author

% XXX Review the order of authors

\def\Authors{Alexandre Abraham\,$^{1,2,*}$, Philippe Gervais\,$^{1,2}$, Fabian
Pedregosa\,$^{1,2}$, Andreas Muller, Jean Kossaifi, Michael Eickenberg, Alexandre Gramfort, Bertrand
Thirion\,$^{1,2}$ and Ga\"el Varoquaux\,$^{1,2}$}
% Affiliations should be keyed to the author's name with superscript numbers and be listed as follows: Laboratory, Institute, Department, Organization, City, State abbreviation (USA, Canada, Australia), and Country (without detailed address information such as city zip codes or street names).
% If one of the authors has a change of address, list the new address below the correspondence details using a superscript symbol and use the same symbol to indicate the author in the author list.
\def\Address{
    $^{1}$Parietal Team, INRIA Saclay-\^{I}le-de-France, Saclay, France\\
    $^{2}$Neurospin, I\textsuperscript{2}BM, DSV, CEA, 91191 Gif-Sur-Yvette, France}

% The Corresponding Author should be marked with an asterisk
% Provide the exact contact address (this time including street name and city zip code) and email of the corresponding author
\def\corrAuthor{Alexandre Abraham}
\def\corrAddress{Parietal Team, INRIA Saclay-\^{I}le-de-France, Saclay, France}
\def\corrEmail{alexandre.abraham@inria.fr}

% \color{FrontiersBlue} Is the blue color, used in the Journal name, in the title, and the names of the sections


% Example of table from template

% \begin{table}[!t]
% \processtable{Resolution Requirements for the figures\label{Tab:01}}
% {\begin{tabular}{lllll}\toprule
% Image Type & Description & Format & Color Mode & Resolution\\\midrule
% Line Art & An image composed of lines and text,  & TIFF, EPS, JPEG & RGB, Bitmap & 900 - 1200 dpi\\
%            & which does not contain tonal or shaded areas.& & &\\
%            Halftone & A continuous tone photograph, which contains no text. & TIFF, EPS, JPEG & RGB, Grayscale & 300 dpi\\
% Combination & Image contains halftone + text or line art elements. & TIFF, EPS, JPEG & RGB,Grayscale & 600 - 900 dpi\\\botrule
% \end{tabular}}{This is a footnote}
% \end{table}


% Figures

% \textbf{Figure 1.}{ Enter the caption for your figure here.  Repeat as  necessary for each of your figures.}\label{fig:01}
% Don't add the figures in the LaTeX files, please upload them when submitting the article. Frontiers will add the figures at the end of the provisional pdf.



\begin{document}
\onecolumn
\firstpage{1}

\title[Machine Learning for Neuroimaging with Scikit-Learn]{Machine Learning for Neuroimaging with Scikit-Learn}
\author[\firstAuthorLast ]{\Authors}
\address{}
\correspondance{}
\editor{}
\topic{Research Topic}

\maketitle
\begin{abstract}

\section{}
Statistical learning methods are increasingly used to perform
neuroimaging analysis. Their main virtue for this type of application
is their ability to model high-dimensional datasets, e.g.\ multivariate
analysis of activation images, or capturing inter-subject variability.
Supervised learning is typically used in “decoding” setting to relate
brain images to behavioral or clinical observations, while
unsupervised learning is typically used to uncover hidden structure in
sets of images (e.g.\ resting state functional MRI) or to find
sub-populations in large cohorts of subjects. By considering
functional neuroimaging use cases, we illustrate how the Scikit-learn,
a Python machine learning library, can be used to perform some key
analysis steps. Scikit-learn contains a large set of statistical
learning algorithms, both supervised and unsupervised, that can be applied
to neuroimaging data after a proper preprocessing. Combined with other
Python libraries, neuroimaging data can be loaded, processed and the results
can be visualised easily.



\tiny
% XXX Fix keywords
%All article types: you may provide up to 8 keywords; at least 5 are mandatory.
\section{Keywords:} Machine learning, Statistical Learning, Neuroimaging, Scikit-learn, Python
\end{abstract}


\section{Introduction}


\subsection{Scientific Python and neuroimaging ecosystem}

\subsubsection{Scipy and Numpy}
\subsubsection{nibabel}



\subsubsection{nipy}
\subsubsection{scikit-learn}



% Here we present the Python neuroimaging ecosystem : scikit­learn, nibabel,
% nipy.

\section{Scikit-learn concepts}

% Present the underlying concepts of scikit­learn: estimator,­ data
% representation, transformer...

\subsection{Estimator}

\subsection{Data representation}

Explain that the scikit process 2D data. This is an introduction to masking.

\subsection{Transformer}

\subsection{Cross validation}

\alex{It seems more right to me to put it in this part}


\section{From MR volumes to a data matrix}

As any domain specific data, MR volumes holds particular properties.
Understanding them is crucial to be sure to make proper use of the data.

\[
    \begin{bmatrix}
        r_x & 0   & 0   & o_x \\
        0   & r_y & 0   & o_y \\
        0   & 0   & r_z & o_z \\
        0   & 0   & 0   & 1   \\
    \end{bmatrix}
    \begin{bmatrix}
        x \\
        y \\
        z \\
        1 \\
    \end{bmatrix}
\]

\subsection{Data Preparation}
    % or Signal Processing ?

At this point, we suppose that standard preprocessings have been applied to the
data. They should be registrated on a common template (MNI for example).
However, data is not yet ready to be processed by the scikit-learn. In
fact, preprocessed data may have different shapes. Moreover, it is essential to
get rid of some remaining scanner artefacts and individual trends.

\subsubsection{Detrending}

Detrending is an essential step when dealing with fMRI data. It removes a
best-fit linear trend (in the least square sense) over the time series of each
voxel. It is obviously needed when you want to study the correlation between
features.



\subsection{Resampling}

Resampling consists in changing the shape of the data. This is typically needed
when dealing when data coming from an heterogenous dataset, as the shape depends
on acquisition parameters.

Practically, resampling is an interpolation and thus may alterate the integrity
of the data. That is why it should be used carefully. Oversampling (increasing
data resolution) leads to higher memory consumption and computation resources.
Downsampling is commonly used to reduce the size of the data we want to process.

Typical sizes are 2mm or 3mm resolutions, but the spread of high field MR
scanner tends to lower these values.


\begin{itemize}
    \item Removing confounds is necessary for some treatments
\end{itemize}

\subsection{Signal cleaning}

\begin{itemize}
    \item Remove high frequency (scanner artefacts)
\end{itemize}

% I think that we should present the challenges about NI data here. Poor SNR,
% multi subject / session...

\subsection{Masking}

% This   step  turns  the  data  into  the  scikit­learn   compliant  formant
% n_features  x  n_samples.  We  may speak of connectivity graph that allow to
% integrate the 3D structure of the data in some algorithms.

\subsubsection{From 4-dimensional image to 2-dimensional array}

Neuroimaging data are represented in 4 dimensions: 3 dimensions for the scans,
which are positioned in a coordinate space, and one dimension for the time.
Scikit-learn algorithms, on the other hand, only accept 2-dimensional data: one
dimension for the features and one for the samples.\\

Consequently, in order to use neuroimaging data in the scikit-learn, a
conversion is needed. The most simple way to achieve that would be to
\emph{flatten} the 3D scans into a 1D array. However, we know that not every
voxels in a neuroimaging scan is useful. In particular, outter-brain voxels are
of no use and, worse, they can bring spurious noise and scanner artefacts (such
as ghosts).\\

To sort out voxels of interest, we will have to apply a mask on the data. Most
of public datasets provide a mask, come of them even provide several, isolating
different functional or anatomical brain regions. \alex{ref to Haxby}

\includegraphics[width=.5\linewidth]{img/niimgs.jpg}

\alex{Should tell here that some algorithms, like logistic regression, do not
like colinear features.}

\subsubsection{Automatically computing a mask}

The simplest strategy to compute a mask is a binarization by a selected threshold.
Due to the nature of the neuroimaging data, there exists some strategy to choose
this threshold in order to obtain a decent segmentation.

\alex{There is a reference for the method used in Nisl. We should put it there
and in the code. Add a figure with an histogram to illustrate.}

Multi subject computation is simply done by intersecting subjects maps
relatively to a chose threshold.

\subsubsection{Conserving geometrical structure}

Applying a mask on the data obviously remove the 3-dimensional structure of the
data. However, some algorithms, like the Ward, need this structural information
to run.

\begin{itemize}
    \item Speak about connectivity graphs / adjacency matrices 
\end{itemize}

\section{Decoding}

The process of predicting behavioral or comportamental data from fMRI scan is
called decoding.



\subsection{SVM}

\begin{itemize}
    \item Precise that we use ANOVA to reduce dimensionality
    \item Introduce the pipeline
    \item Introduce Haxby dataset
\end{itemize}


\subsection{Searchlight}

\begin{itemize}
    \item Present the Searchlight problem
    \item Say it is less a pain to implement thanks to scikit-learn bricks
        (estimator and cross\_val). Plus it is easily customizable.
\end{itemize}

\subsection{Classification of M/EEG sensor space data}


\subsection{Orthogonal Matching Pursuit}




\includegraphics[width=.5\linewidth]{img/logistic_l1_scores.png}

\section{Encoding}

\alex{After talking with Michael, he told me that he could make a fairly simple
    example for encoding, which I think is a plus for the paper. The example will
    be integrated in Nisl.}




\section{Functional Connectivity}

\alex{Should we speak of correlation matrices to represent interaction between
regions?}

\subsection{Independent Component Analysis (ICA)}

\subsubsection{Intuition}

ICA is a blind source separation method. Its principle is to separate a
multivariate signal into several components by maximizing their non-gaussianity.
A typical example is the \emph{cocktail party problem} where ICA separates the
voices of people using signal from several mikes.

It is historically the reference method to extract networks from resting state
fMRI \cite{biswal1999}.

\subsubsection{Application}

\includegraphics[width=.5\linewidth]{img/plot_canica_resting_state_17.png}

\subsection{Clustering}

Make  an  example  with   Ward   Clustering.  Indicate  then   that  other
algorithms  can  be  used  such  as
KMeans and Spectral clustering and only give results.

We use a PCA here to reduce dimensionality.

Bonus: may be used as dimensionality reduction


\includegraphics[width=.3\linewidth]{img/plot_rest_clustering_1.png}
\includegraphics[width=.3\linewidth]{img/plot_rest_clustering_2.png}
\includegraphics[width=.3\linewidth]{img/plot_rest_clustering_3.png}

\section*{Disclosure/Conflict-of-Interest Statement}
%All relationships financial, commercial or otherwise that might be perceived
%by the academic community as representing a potential conflict of interest
%must be described. If no such relationship exists, authors will be asked to
%declare that the research was conducted in the absence of any commercial or
%financial relationships that could be construed as a potential conflict of
%interest.
The authors declare that the research was conducted in the absence of any
commercial or financial relationships that could be construed as a potential
conflict of interest.

\section*{Acknowledgement} Text Text Text Text Text Text  Text Text Text Text
Text Text Text Text  Text Text Text Text Text Text Text Text Text  Text Text
Text.

\paragraph{Funding\textcolon} Text Text Text Text Text Text  Text Text.

\section*{Supplemental Data} Text Text Text Text Text Text  Text Text Text Text
Text Text Text Text Text  Text Text Text Text Text Text Text Text Text  Text
Text Text.

\bibliographystyle{frontiersinSCNS} % for Science articles
\bibliography{biblio}

\end{document}
